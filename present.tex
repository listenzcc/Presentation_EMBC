\documentclass[aspectratio=169]{beamer}

\usepackage{beamerthemeshadow}
\usepackage[utf8]{inputenc}

\usepackage{dblfloatfix}

\usetheme{Warsaw}
% \usecolortheme{spruce}

% \setbeamertemplate{footline}[frame number]

\title[Temporal Dynamics on RSVP using MEG]{Temporal Dynamics on Decoding Target Stimuli in Rapid Serial Visual Presentation using Magnetoencephalography}
\author[Chuncheng Zhang]{Chuncheng Zhang\inst{1}, Shuang Qiu\inst{1}, Shengpei Wang\inst{1}, Wei Wei\inst{1}, Huiguang He\inst{1}}
\institute[IACAS]
{
  \inst{1}
  Research Center for Brain-inspired Intelligence, Institute of Automation, Chinese Academy of Science, Beijing, China.
}

\date{\today}

\begin{document}

\begin{frame}[plain]
    \titlepage
\end{frame}

\begin{frame}[plain]
    \frametitle{Table of Contents}
    \tableofcontents[hideallsubsections]
\end{frame}

\begin{frame}
    \frametitle{Detail of Contents (Delete)}
    \tableofcontents
\end{frame}

% ------------------------------------------------------------------------
\section{Introduction}
\begin{frame}[plain]
    \frametitle{Table of Contents}
    \tableofcontents[currentsection, hideothersubsections]
\end{frame}

\begin{frame}
    \frametitle{Introduction}
    Background

    \begin{itemize}
        \item Rapid serial visual presentation (RSVP) has been widely used in brain-computer interface (BCI) as a high efficient paradigm.
        \item RSVP-BCI has been applied in many areas such as data categorization, face recognition,speller and website evaluation.
    \end{itemize}

\end{frame}

\begin{frame}
    \frametitle{Introduction}
    Motivation

    \begin{itemize}
        \item Little has been known about the temporal dynamics of the neural activity that triggered by target stimuli in RSVP.
        \item Besides the successful engineering applications of RSVP-BCI, the underlying neural activity is still unclear.
    \end{itemize}
\end{frame}

\begin{frame}
    \frametitle{Introduction}
    This work

    \begin{itemize}
        \item The temporal dynamic of target event-related responses in a static RSVP paradigm was investigated using paired structural MRI and MEG signal with different frequency bands.
        \item The MVPA was applied on MEG epoch responses to estimate the decoding power dynamic.
    \end{itemize}

\end{frame}

% ------------------------------------------------------------------------
\section{Experiment and Methods}
\begin{frame}[plain]
    \frametitle{Table of Contents}
    \tableofcontents[currentsection, hideothersubsections]
\end{frame}

\subsection{Task Design}
\begin{frame}
    \frametitle{Task Design}
    \begin{itemize}
        \item Recruited 10 college students (7 males and 3 females, aged 23.79±3.6).
        \item During a block, 100 pictures were shown to the subject in random ordered sequences at a rate of 10 Hz.
        \item The chance of target pictures (odd ball) was set to 4\%.
    \end{itemize}

    \begin{columns}
        \column{0.5\textwidth}
        \begin{figure}[h]
            \centering
            \includegraphics[scale=0.15]{figures/raws/image_0006.jpg}
            \caption{Target picture}
        \end{figure}

        \column{0.5\textwidth}
        \begin{figure}[h]
            \centering
            \includegraphics[scale=0.15]{figures/raws/image_0014.jpg}
            \caption{Non-target picture}
        \end{figure}

    \end{columns}
\end{frame}

\subsection{MEG and MRI acquisition}
\begin{frame}
    \frametitle{MEG and MRI acquisition}

    \begin{itemize}
        \item MEG data were scanned with a whole-head CTF MEG system with 272 channels at the MEG Center of Institute of Biophysics, Chinese Academy of Sciences.
        \item MRI data were scanned with a 3.0 T MRI scanner.
        \item The MRI Center of Institute of Biophysics, Chinese Academy of Sciences.
    \end{itemize}

    \begin{columns}
        \column{0.5\textwidth}
        \begin{figure}[h]
            \centering
            \includegraphics[scale=0.15]{figures/raws/sensors.png}
            \caption{272 MEG sensors}
        \end{figure}

        \column{0.5\textwidth}
        \begin{figure}[h]
            \centering
            \includegraphics[scale=0.12]{figures/raws/75yo_male.png}
            \caption{MRI image}
        \end{figure}

    \end{columns}

\end{frame}

% ------------------------------------------------------------------------
\subsection{MEG Preprocessing}
\begin{frame}
    \frametitle{MEG Preprocessing}

    \begin{columns}
        \column{0.5\textwidth}
        \begin{itemize}
            \item The MEG data were preprocessed using \hyperlink{https://mne.tools/stable/index.html}{\emph{MNE}} software.
            \item Suppressing artificial noise using ICA method.
                  The artificial sources were zeroed out from raw data.
            \item The bands used in this research were Delta, Theta, Alpha bands, and two custom bands: U07 and U30 band.
        \end{itemize}

        \column{0.5\textwidth}

        \begin{table}[t]
            \caption{Filter bands}
            \begin{tabular}{|l|l|}
                \hline
                \textbf{Filter Name} & \textbf{Freq band} \\
                \hline
                \hline
                Delta                & $1 - 4 Hz$         \\
                Theta                & $4 - 7 Hz$         \\
                Alpha                & $8 - 12 Hz$        \\
                \hline
                \hline
                U07                  & $0.1 - 7 Hz$       \\
                U30                  & $0.1 - 30 Hz$      \\
                \hline
            \end{tabular}
        \end{table}

    \end{columns}
\end{frame}

\subsection{MVPA}
\begin{frame}
    \frametitle{MVPA}
    \begin{columns}
        \column{0.6\textwidth}
        \begin{itemize}
            \item Feature extraction was applied to training data, using xDAWN algorithm.
                  Number of components was set as $6$.
            \item Support Vector Machine (SVM) was applied as classifier.
            \item The MVPA was applied in a 10-folder cross-validation protocol.
                  In each folder, we use one run as testing data and others as training data.
        \end{itemize}

        \column{0.5\textwidth}
        \begin{figure}[h]
            \centering
            \includegraphics[scale=0.25]{figures/cv.png}
            \caption{Cross validation process}
        \end{figure}

    \end{columns}
\end{frame}

\subsection{Cortical Neuronal Activation Estimation}
\begin{frame}
    \frametitle{Cortical Neuronal Activation Estimation}

    \begin{itemize}
        \item [Surfaces] The subject-specific cortical surfaces were build based on the MRI data using \hyperlink{https://surfer.nmr.mgh.harvard.edu/}{\emph{freesurfer}} software.
        \item [Model] A forward model was calculated to project the MEG data into cortical surfaces using the \emph{'oct6'} space.
    \end{itemize}

    \begin{columns}
        \column{0.3\textwidth}
        \begin{figure}[h]
            \centering
            \includegraphics[scale=0.15]{figures/raws/OIP.jpg}
            \caption{Surface}
        \end{figure}

        \column{0.7\textwidth}
        \begin{table}
            \caption{oct6 space}
            \begin{tabular}{|c|c|}
                \hline
                \textbf{Spacing}                 & \textbf{Value} \\
                \hline
                \hline
                Sources per hemisphere           & $4098$         \\
                Source spacing $(mm)$            & $4.9$          \\
                Surface area per source $(mm^2)$ & $24.0$         \\
                \hline
            \end{tabular}
        \end{table}

    \end{columns}

\end{frame}

% ------------------------------------------------------------------------
\section{Results and Discussion}
\begin{frame}[plain]
    \frametitle{Table of Contents}
    \tableofcontents[currentsection, hideothersubsections]
\end{frame}

\subsection{MEG Signal Visualization}
\begin{frame}
    \frametitle{MEG Signal Visualization}
    We plot the evoked response of target pictures:

    \begin{columns}
        \column{0.6\textwidth}
        \begin{figure}[h]
            \centering
            \includegraphics[scale=0.4]{figures/raws/joint_U07.png}
            \caption{Evoked of U07 band}
        \end{figure}

        \column{0.4\textwidth}
        \begin{figure}[!h]
            \centering
            \includegraphics[scale=0.3]{figures/raws/joint_U30.png}
            \caption{Evoked of U30 band}
        \end{figure}

    \end{columns}

\end{frame}

\subsection{MVPA Scores}
\begin{frame}
    \frametitle{MVPA Scores}
    The band of \emph{U07} yields highest classification scores:
    \begin{figure}[h]
        \centering
        \includegraphics[scale=0.4]{figures/Accs_all.png}
        \caption{Scores across different bands}
    \end{figure}
\end{frame}

\subsection{MVPA Scores}
\begin{frame}
    \frametitle{MVPA Scores}
    The sources in temporal resolution:
    \begin{figure}[h]
        \centering
        \includegraphics[scale=0.4]{figures/Accs_time.png}
        \caption{Scores in temporal resolution}
    \end{figure}
\end{frame}

\subsection{Cortical Neuronal Activation}
\begin{frame}
    \frametitle{Cortical Neuronal Activation}
    The activity in surfaces:
    \begin{figure}[h]
        \centering
        \includegraphics[scale=0.3]{figures/mean.png}
        \caption{Activity in surfaces}
    \end{figure}
\end{frame}

\subsection{Cortical Neuronal Activation}
\begin{frame}
    \frametitle{Cortical Neuronal Activation}
    The activity in surfaces:
    \begin{figure}[h]
        \centering
        \includegraphics[scale=0.35]{figures/source_both.png}
        \caption{Activity in surfaces}
    \end{figure}
\end{frame}

% ------------------------------------------------------------------------
\section{Conclusion and Acknowledgements}
\begin{frame}[plain]
    \frametitle{Table of Contents}
    \tableofcontents[currentsection, hideothersubsections]
\end{frame}

\begin{frame}
    \frametitle{Conclusion}
    \begin{itemize}
        \item The temporal dynamic of target event-related responses in a static RSVP paradigm was investigated using MEG signal with different frequency bands.
        \item The MVPA results showed that the \emph{U07} band signals $(0.1-7 Hz)$ yielded highest decoding accuracy, and further uncover the decoding power dynamic reached its peak at around $0.4$ second after target stimuli onset.
        \item The cortical neuronal activation identified the target stimuli triggered regions, like \emph{bilateral parahippocampal cortex}, \emph{precentral gyrus} and \emph{insula cortex}.
    \end{itemize}
\end{frame}

\begin{frame}
    \frametitle{Acknowledgements}
    Big thanks
\end{frame}

\end{document}