\documentclass[aspectratio=169]{beamer}

\usepackage{beamerthemeshadow}
\usepackage[utf8]{inputenc}

\usepackage{dblfloatfix}

\usetheme{Warsaw}
% \usecolortheme{spruce}

% \setbeamertemplate{footline}[frame number]

\title[Temporal Dynamics on RSVP using MEG]{Temporal Dynamics on Decoding Target Stimuli in Rapid Serial Visual Presentation using Magnetoencephalography}
\author[Chuncheng Zhang]{Chuncheng Zhang\inst{1}, Shuang Qiu\inst{1}, Shengpei Wang\inst{1}, Wei Wei\inst{1}, Huiguang He\inst{1}}
\institute[IACAS]
{
  \inst{1}
  Research Center for Brain-inspired Intelligence, Institute of Automation, Chinese Academy of Science, Beijing, China.
}

\date{\today}

\begin{document}

\begin{frame}[plain]
    \titlepage
\end{frame}

\begin{frame}[plain]
    \frametitle{Table of Contents}
    \tableofcontents[hideallsubsections]
\end{frame}

% \begin{frame}
%     \frametitle{Detail of Contents (Delete)}
%     \tableofcontents
% \end{frame}

% ------------------------------------------------------------------------
\section{Introduction}

\begin{frame}[plain]
    \frametitle{Table of Contents}
    \tableofcontents[currentsection, hideothersubsections]
\end{frame}

\subsection{Background}

\begin{frame}
    \frametitle{Background}

    \begin{itemize}
        \item Rapid serial visual presentation (RSVP) has been widely used in brain-computer interface (BCI) as a high efficient paradigm.
        \item RSVP uses odd-ball effect to detect target stimuli in rapid serial.
        \item RSVP-BCI has been applied in many areas such as data categorization, face recognition, speller and website evaluation.
    \end{itemize}

    \begin{figure}[h]
        \centering
        \includegraphics[height=0.2\textheight]{figures/planeringspresHT13-Oddball-sekvens.jpg}
        \caption{Paradigm of RSVP}
    \end{figure}

\end{frame}

\subsection{Motivation}

\begin{frame}
    \frametitle{Motivation}

    \begin{itemize}
        \item RSVP-BCI is largely relay on the accuracy of event-related response using single trail.
        \item The temporal dynamics of the discriminate power is still unclear.
        \item The underlying neural activity is still unclear.
    \end{itemize}

    \pause

    \begin{block}{Motivation}

        \begin{itemize}
            \item Investigate \textbf{discriminate power dynamic}.
            \item Investigate \textbf{neural activity}.
        \end{itemize}

    \end{block}

\end{frame}

\subsection{This work}

\begin{frame}
    \frametitle{This work}

    \begin{itemize}
        \item Perform RSVP experiment with MEG scanning and paired MRI imaging.
        \item Investigate the temporal dynamic of discriminate power using MVPA method under different frequency bands.
        \item Investigate the underlying neural activity using paired analysis of MEG and MRI.
    \end{itemize}

\end{frame}

% ------------------------------------------------------------------------
\section{Experiment and Methods}

\begin{frame}[plain]
    \frametitle{Table of Contents}
    \tableofcontents[currentsection, hideothersubsections]
\end{frame}

\subsection{Experiment Design}

\begin{frame}
    \frametitle{Experiment Design}

    \begin{itemize}
        \item Recruited 10 college students ($7$ males and $3$ females, aged $23.79 \pm 3.6$).
        \item Each participant perform $10$ blocks, during a block, $100$ street view pictures were shown in random order at $10 Hz$.
        \item The chance of target pictures (odd ball) was set to 4\%.
    \end{itemize}

    \begin{columns}
        \column{0.3\textwidth}

        \begin{figure}[h]
            \centering
            \includegraphics[height=0.35\textheight]{figures/raws/image_0006.jpg}
            \caption{Target picture}
        \end{figure}

        \column{0.3\textwidth}

        \begin{figure}[h]
            \centering
            \includegraphics[height=0.35\textheight]{figures/raws/image_0014.jpg}
            \caption{Non-target picture}
        \end{figure}

        \column{0.3\textwidth}

        \begin{figure}[h]
            \centering
            \includegraphics[height=0.35\textheight]{figures/design.png}
            \caption{Time line}
        \end{figure}

    \end{columns}

\end{frame}

\subsection{MEG and MRI acquisition}

\begin{frame}
    \frametitle{MEG and MRI acquisition}

    \begin{itemize}
        \item MEG data were scanned with a whole-head CTF MEG system with 272 channels.
        \item MRI data were scanned with a $3.0 T$ MRI scanner.
        \item The MRI Center of Institute of Biophysics, Chinese Academy of Sciences.
    \end{itemize}

    \begin{columns}
        \column{0.3\textwidth}

        \begin{figure}[h]
            \centering
            \includegraphics[height=0.4\textheight]{figures/raws/sensors.png}
            \caption{272 MEG sensors}
        \end{figure}

        \column{0.3\textwidth}

        \begin{figure}[h]
            \centering
            \includegraphics[height=0.4\textheight]{figures/raws/75yo_male.png}
            \caption{MRI image}
        \end{figure}

        \column{0.3\textwidth}

        \begin{figure}[h]
            \centering
            \includegraphics[height=0.4\textheight]{figures/sphx_glr_plot_eeg_no_mri_001.png}
            \caption{Alignment}
        \end{figure}

    \end{columns}

\end{frame}

\subsection{Cortical Neuronal Activation Estimation}

\begin{frame}
    \frametitle{Cortical Neuronal Activation Estimation}

    \begin{itemize}
        \item The subject-specific cortical surfaces were build based on the MRI data using \hyperlink{https://surfer.nmr.mgh.harvard.edu/}{\emph{freesurfer}} software.
        \item A forward model was calculated to project the MEG data into cortical surfaces using the \emph{'oct6'} space.
    \end{itemize}

    \begin{columns}
        \column{0.4\textwidth}

        \begin{figure}[h]
            \centering
            \includegraphics[width=0.7\textwidth]{figures/raws/OIP.jpg}
            \caption{Surface}
        \end{figure}

        \column{0.6\textwidth}

        \begin{table}
            \renewcommand{\arraystretch}{1.2}
            \caption{oct6 space}

            \begin{tabular}{|c|c|}
                \hline
                \textbf{Spacing}                 & \textbf{Value} \\
                \hline
                \hline
                Sources per hemisphere           & $4098$         \\
                Source spacing $(mm)$            & $4.9$          \\
                Surface area per source $(mm^2)$ & $24.0$         \\
                \hline
            \end{tabular}

        \end{table}

    \end{columns}

\end{frame}

\subsection{MEG Preprocessing}

\begin{frame}
    \frametitle{MEG Preprocessing}

    \begin{columns}
        \column{0.5\textwidth}

        \begin{itemize}
            \item The MEG data were preprocessed using \hyperlink{https://mne.tools/stable/index.html}{\emph{MNE}} software.
            \item Suppressing artificial noise using ICA method.
                  The artificial sources were zeroed out from raw data.
            \item The bands used in this research were Delta, Theta, Alpha bands, and two custom bands: U07 and U30 band.
        \end{itemize}

        \column{0.5\textwidth}

        \begin{table}[t]
            \renewcommand{\arraystretch}{1.2}
            \caption{Filter bands}

            \begin{tabular}{|l|l|}
                \hline
                \textbf{Filter Name} & \textbf{Freq band} \\
                \hline
                \hline
                Delta                & $1 - 4 Hz$         \\
                Theta                & $4 - 7 Hz$         \\
                Alpha                & $8 - 12 Hz$        \\
                \hline
                \hline
                U07                  & $0.1 - 7 Hz$       \\
                U30                  & $0.1 - 30 Hz$      \\
                \hline
            \end{tabular}

        \end{table}

    \end{columns}

\end{frame}

\subsection{MVPA}

\begin{frame}
    \frametitle{MVPA}

    \begin{columns}
        \column{0.6\textwidth}

        \begin{itemize}
            \item Feature extraction was applied to training data, using xDAWN algorithm.
                  Number of components was set as $6$.
            \item Support Vector Machine (SVM) with RBF kernel was used as classifier.
            \item The MVPA was applied in a 10-folder cross-validation protocol.
                  In each folder, we use one run as testing data and others as training data.
        \end{itemize}

        \column{0.5\textwidth}

        \begin{figure}[h]
            \centering
            \includegraphics[width=1.0\textwidth]{figures/cv.png}
            \caption{Cross validation process}
        \end{figure}

    \end{columns}

\end{frame}

\section{Results and Discussion}

\begin{frame}[plain]
    \frametitle{Table of Contents}
    \tableofcontents[currentsection, hideothersubsections]
\end{frame}

\subsection{MEG Signal Visualization}

\begin{frame}
    \frametitle{MEG Signal Visualization}
    We plot the evoked response of target pictures:

    \begin{columns}
        \column{0.6\textwidth}

        \begin{figure}[h]
            \centering
            \includegraphics[width=0.9\textwidth]{figures/raws/joint_U07.png}
            \caption{Evoked of U07 band}
        \end{figure}

        \column{0.4\textwidth}

        \begin{figure}[!h]
            \centering
            \includegraphics[width=0.9\textwidth]{figures/raws/joint_U30.png}
            \caption{Evoked of U30 band}
        \end{figure}

    \end{columns}

\end{frame}

\subsection{Cortical Neuronal Activation}

\begin{frame}
    \frametitle{Cortical Neuronal Activation}
    The activity in surfaces:

    \begin{figure}[h]
        \centering
        \includegraphics[width=1.0\textwidth]{figures/mean.png}
        \caption{Activity in surfaces}
    \end{figure}

\end{frame}

\subsection{Cortical Neuronal Activation}

\begin{frame}
    \frametitle{Cortical Neuronal Activation}
    The activity in surfaces:

    \begin{figure}[h]
        \centering
        \includegraphics[width=1.0\textwidth]{figures/source_both.png}
        \caption{Activity in surfaces}
    \end{figure}

\end{frame}

\subsection{MVPA Scores}

\begin{frame}
    \frametitle{MVPA Scores}
    We use following scores to measure the discriminating power:

    \begin{table}
        \renewcommand{\arraystretch}{1.5}
        \caption{MVPA Scores}

        \begin{tabular}{|c|c|c|}
            \hline
            \textbf{Name} & \textbf{Meaning}                       & \textbf{Higher means}          \\
            \hline
            \hline
            Recall        & $ \frac{TP}{TP+FN} $                   & More target being selected     \\
            \hline
            Precision     & $ \frac{TP}{TP+FP} $                   & Less non-target being selected \\
            \hline
            f1            & $ \frac{2 \cdot P \cdot R}{\ P+R \ } $ & Better target selection        \\
            \hline
            Accuracy      & $ \frac{TP+FP}{TP+TN+FP+FN} $          & Higher overall accuracy        \\
            \hline
        \end{tabular}

    \end{table}

\end{frame}

\begin{frame}
    \frametitle{MVPA Scores}
    The band of \emph{U07} yields highest classification scores:

    \begin{figure}[h]
        \centering
        \includegraphics[width=0.9\textwidth]{figures/Accs_all.png}
        \caption{Scores across different bands}
    \end{figure}

\end{frame}

\subsection{MVPA Scores}

\begin{frame}
    \frametitle{MVPA Scores}
    The sources in temporal resolution:

    \begin{figure}[h]
        \centering
        \includegraphics[width=0.9\textwidth]{figures/Accs_time.png}
        \caption{Scores in temporal resolution}
    \end{figure}

\end{frame}

% ------------------------------------------------------------------------
\section{Conclusion and Acknowledgements}

\begin{frame}[plain]
    \frametitle{Table of Contents}
    \tableofcontents[currentsection, hideothersubsections]
\end{frame}

\subsection{Conclusion}

\begin{frame}
    \frametitle{Conclusion}
    \begin{itemize}
        \item The temporal dynamic of target event-related responses in a static RSVP paradigm was investigated using MEG signal with different frequency bands.
        \item The MVPA results showed that the \emph{U07} band signals $(0.1-7 Hz)$ yielded highest decoding accuracy, and further uncover the decoding power dynamic reached its peak at around $0.4$ second after target stimuli onset.
        \item The cortical neuronal activation identified the regions activated by target pictures, like \emph{bilateral parahippocampal cortex}, \emph{precentral gyrus} and \emph{insula cortex}.
    \end{itemize}
\end{frame}

\subsection{Acknowledgements}
\begin{frame}
    \frametitle{Acknowledgements}
    Appreciate the Grants for supporting the work:
    \begin{itemize}
        \item The National Natural Science Foundation of China Grant 61976209.
        \item The National Natural Science Foundation of China Grant 61906188.
        \item The National Natural Science Foundation of China Grant 81701785.
        \item The Strategic Priority Research Program of CAS under Grant XDB32040200.
        \item The China Postdoctoral Science Foundation 2019M650893.
    \end{itemize}
\end{frame}

\end{document}