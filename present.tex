\documentclass{beamer}

\usepackage{beamerthemeshadow}
\usepackage[utf8]{inputenc}

\usetheme{Warsaw}
% \usecolortheme{spruce}

% \setbeamertemplate{footline}[frame number]

\title[Temporal Dynamics on RSVP using MEG]{Temporal Dynamics on Decoding Target Stimuli in Rapid Serial Visual Presentation using Magnetoencephalography}
\author[Chuncheng Zhang]{Chuncheng Zhang\inst{1}, Shuang Qiu\inst{1}, Shengpei Wang\inst{1}, Wei Wei\inst{1}, Huiguang He\inst{1}}
\institute[IACAS]
{
  \inst{1}
  Research Center for Brain-inspired Intelligence, Institute of Automation, Chinese Academy of Science, Beijing, China.
}

\date{\today}

\begin{document}

\begin{frame}[plain]
    \titlepage
\end{frame}

\begin{frame}[plain]
    \frametitle{Table of Contents}
    \tableofcontents[hideallsubsections]
\end{frame}

\begin{frame}
    \frametitle{Detail of Contents (Delete)}
    \tableofcontents
\end{frame}

% ------------------------------------------------------------------------
\section{Introduction}
\begin{frame}[plain]
    \frametitle{Table of Contents}
    \tableofcontents[currentsection, hideothersubsections]
\end{frame}

\begin{frame}
    \frametitle{Introduction}
    Background

    Rapid serial visual presentation (RSVP) has been widely used in brain-computer interface (BCI) as a high efficient paradigm [1]. RSVP-BCI has been applied in many areas such as data categorization [2], face recognition [3] ,speller [4] and website evaluation [5].
\end{frame}

\begin{frame}
    \frametitle{Introduction}
    Motivation

    \begin{itemize}
        \item Little has been known about the temporal dynamics of the neural activity that triggered by target stimuli in RSVP.
        \item Besides the successful engineering applications of RSVP-BCI, the underlying neural activity is still unclear.
    \end{itemize}
\end{frame}

\begin{frame}
    \frametitle{Introduction}
    This work

    \begin{itemize}
        \item The temporal dynamic of target event-related responses in a static RSVP paradigm was investigated using paired structural MRI and MEG signal with different frequency bands.
        \item The MVPA was applied on MEG epoch responses to estimate the decoding power dynamic.
    \end{itemize}

\end{frame}

% ------------------------------------------------------------------------
\section{Experiment}
\begin{frame}[plain]
    \frametitle{Table of Contents}
    \tableofcontents[currentsection, hideothersubsections]
\end{frame}

\subsection{Task Design}
\begin{frame}
    \frametitle{Task Design}

    \begin{itemize}
        \item [Subjects] Recruited 10 college students (7 males and 3 females, aged 23.79±3.6).
        \item [Design] During a block, 100 pictures were shown to the subject in random ordered sequences at a rate of 10 Hz.
        \item [Ratio] The chance of target pictures (odd ball) was set to 4\%.
    \end{itemize}

\end{frame}

\subsection{MEG and MRI acquisition}
\begin{frame}
    \frametitle{MEG and MRI acquisition}

    \begin{itemize}
        \item [MEG] MEG data were scanned with a whole-head CTF MEG system with 272 channels (MISL-CTF DSQ-3500, Vancouver, BC, Canada) at the MEG Center of Institute of Biophysics, Chinese Academy of Sciences.
        \item [MRI] MRI data were scanned with a 3.0 T MRI scanner (Siemens, Germany) at the MRI Center of Institute of Biophysics, Chinese Academy of Sciences.
    \end{itemize}

\end{frame}

% ------------------------------------------------------------------------
\section{Methods}
\begin{frame}[plain]
    \frametitle{Table of Contents}
    \tableofcontents[currentsection, hideothersubsections]
\end{frame}

\subsection{MEG Preprocessing}
\begin{frame}
    \frametitle{MEG Preprocessing}

    \begin{itemize}
        \item [Software] The MEG data were preprocessed using \hyperlink{https://mne.tools/stable/index.html}{\emph{MNE}} software.
        \item [De-noise] Suppressing artificial noise using ICA method. The artificial sources were zeroed out from raw data.
        \item [Filter] The bands used in this research were Delta, Theta , Alpha bands, and two custom bands: U07 and U30 band.
    \end{itemize}

    \begin{table}[t]
        \caption{Filter bands}
        \begin{tabular}{|l|l|}
            \hline
            \textbf{Filter Name} & \textbf{Freq band} \\
            \hline
            Delta                & $1 - 4 Hz$         \\
            Theta                & $4 - 7 Hz$         \\
            Alpha                & $8 - 12 Hz$        \\
            \hline
            U07                  & $0.1 - 7 Hz$       \\
            U30                  & $0.1 - 30 Hz$      \\
            \hline
        \end{tabular}
    \end{table}

\end{frame}

\subsection{MVPA}
\begin{frame}
    \frametitle{MVPA}
    \begin{itemize}
        \item [Feature] Feature extraction was applied to training data, using xDAWN algorithm.
              Number of components was set as $6$.
        \item [Classifier] Support Vector Machine (SVM) was applied as classifier.
        \item [Validation] The MVPA was applied in a 10-folder cross-validation protocol.
              In each folder, we use one run as testing data and others as training data.
    \end{itemize}
\end{frame}

\subsection{Cortical Neuronal Activation Estimation}
\begin{frame}
    \frametitle{Cortical Neuronal Activation Estimation}

    \begin{itemize}
        \item [Surfaces] The subject-specific cortical surfaces were build based on the MRI data using \hyperlink{https://surfer.nmr.mgh.harvard.edu/}{\emph{freesurfer}} software.
        \item [Model] A forward model was calculated to project the MEG data into cortical surfaces using the \emph{'oct6'} space.
    \end{itemize}

\end{frame}

% ------------------------------------------------------------------------
\section{Results and Discussion}
\begin{frame}[plain]
    \frametitle{Table of Contents}
    \tableofcontents[currentsection, hideothersubsections]
\end{frame}

\subsection{MEG Signal Visualization}
\begin{frame}
    \frametitle{MEG Signal Visualization}
    Joint plot of evoked

    U07

    U30
\end{frame}

\subsection{MVPA Scores}
\begin{frame}
    \frametitle{MVPA Scores}
    Accuracy

    F1-score

    Recall ratio
\end{frame}


\subsection{MVPA Scores}
\begin{frame}
    \frametitle{MVPA Scores}
    Scores in temporal resolution
\end{frame}

\subsection{Cortical Neuronal Activation}
\begin{frame}
    \frametitle{Cortical Neuronal Activation}
    Evoked activations
\end{frame}

% ------------------------------------------------------------------------
\section{Conclusion}
\begin{frame}[plain]
    \frametitle{Table of Contents}
    \tableofcontents[currentsection, hideothersubsections]
\end{frame}

\begin{frame}
    \frametitle{Conclusion}

\end{frame}

% ------------------------------------------------------------------------
\section{Acknowledgements}
\begin{frame}[plain]
    \frametitle{Table of Contents}
    \tableofcontents[currentsection, hideothersubsections]
\end{frame}

\begin{frame}
    \frametitle{Acknowledgements}
    Big thanks
\end{frame}

\end{document}